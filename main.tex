\documentclass{article} % For LaTeX2e
\usepackage{iclr2020_conference,times}

% Optional math commands from https://github.com/goodfeli/dlbook_notation.
\input{math_commands.tex}

\usepackage{hyperref}
\usepackage{url}


\title{Which spherical CNN should you use?\\ DeepSphere V2}

% Authors must not appear in the submitted version. They should be hidden
% as long as the \iclrfinalcopy macro remains commented out below.
% Non-anonymous submissions will be rejected without review. 
\author{Michaël Defferrard, Martino Milani, Frédérick Gusset,  Nathanaël Perraudin \thanks{ Use footnote for providing further information about author (webpage, alternative address)---\emph{not} for acknowledging
funding agencies.  Funding acknowledgements go at the end of the paper.} \\
Department of Computer Science\\
Cranberry-Lemon University\\
Pittsburgh, PA 15213, USA \\
\texttt{\{hippo,brain,jen\}@cs.cranberry-lemon.edu} \\
\And
Ji Q. Ren \& Yevgeny LeNet \\
Department of Computational Neuroscience \\
University of the Witwatersrand \\
Joburg, South Africa \\
\texttt{\{robot,net\}@wits.ac.za} \\
\AND
Coauthor \\
Affiliation \\
Address \\
\texttt{email}
}

% The \author macro works with any number of authors. There are two commands
% used to separate the names and addresses of multiple authors: \And and \AND.
%
% Using \And between authors leaves it to \LaTeX{} to determine where to break
% the lines. Using \AND forces a linebreak at that point. So, if \LaTeX{}
% puts 3 of 4 authors names on the first line, and the last on the second
% line, try using \AND instead of \And before the third author name.

\newcommand{\fix}{\marginpar{FIX}}
\newcommand{\new}{\marginpar{NEW}}

%\iclrfinalcopy % Uncomment for camera-ready version, but NOT for submission.
\begin{document}


\maketitle

\begin{abstract}
No
\end{abstract}

\section{Introduction [2 pages]}
Get inspiration from DeepSphere V1

Why are spherical CNNs important? What are the application?
See DeepSphere workshop paper

Desiderata:
* respect the domain
  * rotational equivariance
  * deformation (e.g., on the icosahedron for gauge)
* powerful / generality => isotropic vs anisotropic
* scale: computational cost \& memory usage (feature maps)
* flexible (samplings, irregular), simplicity (implementation)

\begin{table}[h!]
    \centering
    \begin{tabular}{l|c|c|c|c}
         & Scale & Generality& equivariance & Flexibility \\
         \hline
        Cohen & & & &  \\
         \hline
        Gauge & & & &  \\
         \hline
        Esteves & & & &  \\
         \hline
        Jiang & & & &  \\
         \hline
        2D CNN (cubed sphere) & & & &  \\
         \hline
        Ours & & & &  \\
    \end{tabular}
    \caption{Caption}
    \label{tab:my_label}
\end{table}

Present the different spherical CNNs with pro and cons => mostly scale vs generality
* 2D CNN => doesn't respect geometry and equivariance
* Cohen spherical CNN => most general, but doesn't scale
* Cohen gauge => fixes scale, at the price of deformation (sphere => icosahedron)
* Esteves => ?
* Jiang => need a global coordinate system (ok for planets, not projections like cosmo)

\cite{perraudin2019deepsphere}

\section{Method [2 pages]}

* HEALPix: improvement from DeepSphere v1
* equiangular: Renata \& Pascal

\section{Rotation equivariance [2 pages]}
\subsection{Link with the Laplace-Beltrami Operator}

\subsection{Convergence theorem}

\section{Experiments [2 pages]}

\subsection{3D object recognition}

* same perf as other spherical CNNs, but 40 times faster
* compare the two samplings => not much difference

\subsection{cosmo}

other sphercial CNNs cannot scale to 10M pixels (tested on 10k at most)

\subsection{Climate event detection}

* small: better perf than gauge and Jiang. due to icosahedron distortion? to be confirmed
* cannot compare with Mudigonda (16 vs 1 input channel)
* full: we scale => lead to better perf?

\section{Conclusion}

* DeepSphere seems to be in the sweet spot of tradeoffs. Recall the desiderata.
  * Not the most general, but sufficient empirically
  * scales linearly => cannot do better ($n^1.25$ for HEALPix)
  * flexible => independent of the sampsamples can be 
* anisotropy (most general) doesn't help. research question: when can it be useful?

future work:
* when anisotropy needed?
* irregular sampling while respecting the geometry
* beyond the sphere, any manifold


\newpage

\subsubsection*{Author Contributions}
If you'd like to, you may include  a section for author contributions as is done
in many journals. This is optional and at the discretion of the authors.

\subsubsection*{Acknowledgments}
Use unnumbered third level headings for the acknowledgments. All
acknowledgments, including those to funding agencies, go at the end of the paper.


\bibliography{references}
\bibliographystyle{iclr2020_conference}

\newpage

\appendix
\section{Appendix}

\subsection{}



\end{document}
