\documentclass{article} % For LaTeX2e
\usepackage{iclr2020_conference,times}

% Optional math commands from https://github.com/goodfeli/dlbook_notation.
%%%%% NEW MATH DEFINITIONS %%%%%

\usepackage{amsmath,amsfonts,bm}

% Mark sections of captions for referring to divisions of figures
\newcommand{\figleft}{{\em (Left)}}
\newcommand{\figcenter}{{\em (Center)}}
\newcommand{\figright}{{\em (Right)}}
\newcommand{\figtop}{{\em (Top)}}
\newcommand{\figbottom}{{\em (Bottom)}}
\newcommand{\captiona}{{\em (a)}}
\newcommand{\captionb}{{\em (b)}}
\newcommand{\captionc}{{\em (c)}}
\newcommand{\captiond}{{\em (d)}}

% Highlight a newly defined term
\newcommand{\newterm}[1]{{\bf #1}}


% Figure reference, lower-case.
\def\figref#1{figure~\ref{#1}}
% Figure reference, capital. For start of sentence
\def\Figref#1{Figure~\ref{#1}}
\def\twofigref#1#2{figures \ref{#1} and \ref{#2}}
\def\quadfigref#1#2#3#4{figures \ref{#1}, \ref{#2}, \ref{#3} and \ref{#4}}
% Section reference, lower-case.
\def\secref#1{section~\ref{#1}}
% Section reference, capital.
\def\Secref#1{Section~\ref{#1}}
% Reference to two sections.
\def\twosecrefs#1#2{sections \ref{#1} and \ref{#2}}
% Reference to three sections.
\def\secrefs#1#2#3{sections \ref{#1}, \ref{#2} and \ref{#3}}
% Reference to an equation, lower-case.
\def\eqref#1{equation~\ref{#1}}
% Reference to an equation, upper case
\def\Eqref#1{Equation~\ref{#1}}
% A raw reference to an equation---avoid using if possible
\def\plaineqref#1{\ref{#1}}
% Reference to a chapter, lower-case.
\def\chapref#1{chapter~\ref{#1}}
% Reference to an equation, upper case.
\def\Chapref#1{Chapter~\ref{#1}}
% Reference to a range of chapters
\def\rangechapref#1#2{chapters\ref{#1}--\ref{#2}}
% Reference to an algorithm, lower-case.
\def\algref#1{algorithm~\ref{#1}}
% Reference to an algorithm, upper case.
\def\Algref#1{Algorithm~\ref{#1}}
\def\twoalgref#1#2{algorithms \ref{#1} and \ref{#2}}
\def\Twoalgref#1#2{Algorithms \ref{#1} and \ref{#2}}
% Reference to a part, lower case
\def\partref#1{part~\ref{#1}}
% Reference to a part, upper case
\def\Partref#1{Part~\ref{#1}}
\def\twopartref#1#2{parts \ref{#1} and \ref{#2}}

\def\ceil#1{\lceil #1 \rceil}
\def\floor#1{\lfloor #1 \rfloor}
\def\1{\bm{1}}
\newcommand{\train}{\mathcal{D}}
\newcommand{\valid}{\mathcal{D_{\mathrm{valid}}}}
\newcommand{\test}{\mathcal{D_{\mathrm{test}}}}

\def\eps{{\epsilon}}


% Random variables
\def\reta{{\textnormal{$\eta$}}}
\def\ra{{\textnormal{a}}}
\def\rb{{\textnormal{b}}}
\def\rc{{\textnormal{c}}}
\def\rd{{\textnormal{d}}}
\def\re{{\textnormal{e}}}
\def\rf{{\textnormal{f}}}
\def\rg{{\textnormal{g}}}
\def\rh{{\textnormal{h}}}
\def\ri{{\textnormal{i}}}
\def\rj{{\textnormal{j}}}
\def\rk{{\textnormal{k}}}
\def\rl{{\textnormal{l}}}
% rm is already a command, just don't name any random variables m
\def\rn{{\textnormal{n}}}
\def\ro{{\textnormal{o}}}
\def\rp{{\textnormal{p}}}
\def\rq{{\textnormal{q}}}
\def\rr{{\textnormal{r}}}
\def\rs{{\textnormal{s}}}
\def\rt{{\textnormal{t}}}
\def\ru{{\textnormal{u}}}
\def\rv{{\textnormal{v}}}
\def\rw{{\textnormal{w}}}
\def\rx{{\textnormal{x}}}
\def\ry{{\textnormal{y}}}
\def\rz{{\textnormal{z}}}

% Random vectors
\def\rvepsilon{{\mathbf{\epsilon}}}
\def\rvtheta{{\mathbf{\theta}}}
\def\rva{{\mathbf{a}}}
\def\rvb{{\mathbf{b}}}
\def\rvc{{\mathbf{c}}}
\def\rvd{{\mathbf{d}}}
\def\rve{{\mathbf{e}}}
\def\rvf{{\mathbf{f}}}
\def\rvg{{\mathbf{g}}}
\def\rvh{{\mathbf{h}}}
\def\rvu{{\mathbf{i}}}
\def\rvj{{\mathbf{j}}}
\def\rvk{{\mathbf{k}}}
\def\rvl{{\mathbf{l}}}
\def\rvm{{\mathbf{m}}}
\def\rvn{{\mathbf{n}}}
\def\rvo{{\mathbf{o}}}
\def\rvp{{\mathbf{p}}}
\def\rvq{{\mathbf{q}}}
\def\rvr{{\mathbf{r}}}
\def\rvs{{\mathbf{s}}}
\def\rvt{{\mathbf{t}}}
\def\rvu{{\mathbf{u}}}
\def\rvv{{\mathbf{v}}}
\def\rvw{{\mathbf{w}}}
\def\rvx{{\mathbf{x}}}
\def\rvy{{\mathbf{y}}}
\def\rvz{{\mathbf{z}}}

% Elements of random vectors
\def\erva{{\textnormal{a}}}
\def\ervb{{\textnormal{b}}}
\def\ervc{{\textnormal{c}}}
\def\ervd{{\textnormal{d}}}
\def\erve{{\textnormal{e}}}
\def\ervf{{\textnormal{f}}}
\def\ervg{{\textnormal{g}}}
\def\ervh{{\textnormal{h}}}
\def\ervi{{\textnormal{i}}}
\def\ervj{{\textnormal{j}}}
\def\ervk{{\textnormal{k}}}
\def\ervl{{\textnormal{l}}}
\def\ervm{{\textnormal{m}}}
\def\ervn{{\textnormal{n}}}
\def\ervo{{\textnormal{o}}}
\def\ervp{{\textnormal{p}}}
\def\ervq{{\textnormal{q}}}
\def\ervr{{\textnormal{r}}}
\def\ervs{{\textnormal{s}}}
\def\ervt{{\textnormal{t}}}
\def\ervu{{\textnormal{u}}}
\def\ervv{{\textnormal{v}}}
\def\ervw{{\textnormal{w}}}
\def\ervx{{\textnormal{x}}}
\def\ervy{{\textnormal{y}}}
\def\ervz{{\textnormal{z}}}

% Random matrices
\def\rmA{{\mathbf{A}}}
\def\rmB{{\mathbf{B}}}
\def\rmC{{\mathbf{C}}}
\def\rmD{{\mathbf{D}}}
\def\rmE{{\mathbf{E}}}
\def\rmF{{\mathbf{F}}}
\def\rmG{{\mathbf{G}}}
\def\rmH{{\mathbf{H}}}
\def\rmI{{\mathbf{I}}}
\def\rmJ{{\mathbf{J}}}
\def\rmK{{\mathbf{K}}}
\def\rmL{{\mathbf{L}}}
\def\rmM{{\mathbf{M}}}
\def\rmN{{\mathbf{N}}}
\def\rmO{{\mathbf{O}}}
\def\rmP{{\mathbf{P}}}
\def\rmQ{{\mathbf{Q}}}
\def\rmR{{\mathbf{R}}}
\def\rmS{{\mathbf{S}}}
\def\rmT{{\mathbf{T}}}
\def\rmU{{\mathbf{U}}}
\def\rmV{{\mathbf{V}}}
\def\rmW{{\mathbf{W}}}
\def\rmX{{\mathbf{X}}}
\def\rmY{{\mathbf{Y}}}
\def\rmZ{{\mathbf{Z}}}

% Elements of random matrices
\def\ermA{{\textnormal{A}}}
\def\ermB{{\textnormal{B}}}
\def\ermC{{\textnormal{C}}}
\def\ermD{{\textnormal{D}}}
\def\ermE{{\textnormal{E}}}
\def\ermF{{\textnormal{F}}}
\def\ermG{{\textnormal{G}}}
\def\ermH{{\textnormal{H}}}
\def\ermI{{\textnormal{I}}}
\def\ermJ{{\textnormal{J}}}
\def\ermK{{\textnormal{K}}}
\def\ermL{{\textnormal{L}}}
\def\ermM{{\textnormal{M}}}
\def\ermN{{\textnormal{N}}}
\def\ermO{{\textnormal{O}}}
\def\ermP{{\textnormal{P}}}
\def\ermQ{{\textnormal{Q}}}
\def\ermR{{\textnormal{R}}}
\def\ermS{{\textnormal{S}}}
\def\ermT{{\textnormal{T}}}
\def\ermU{{\textnormal{U}}}
\def\ermV{{\textnormal{V}}}
\def\ermW{{\textnormal{W}}}
\def\ermX{{\textnormal{X}}}
\def\ermY{{\textnormal{Y}}}
\def\ermZ{{\textnormal{Z}}}

% Vectors
\def\vzero{{\bm{0}}}
\def\vone{{\bm{1}}}
\def\vmu{{\bm{\mu}}}
\def\vtheta{{\bm{\theta}}}
\def\va{{\bm{a}}}
\def\vb{{\bm{b}}}
\def\vc{{\bm{c}}}
\def\vd{{\bm{d}}}
\def\ve{{\bm{e}}}
\def\vf{{\bm{f}}}
\def\vg{{\bm{g}}}
\def\vh{{\bm{h}}}
\def\vi{{\bm{i}}}
\def\vj{{\bm{j}}}
\def\vk{{\bm{k}}}
\def\vl{{\bm{l}}}
\def\vm{{\bm{m}}}
\def\vn{{\bm{n}}}
\def\vo{{\bm{o}}}
\def\vp{{\bm{p}}}
\def\vq{{\bm{q}}}
\def\vr{{\bm{r}}}
\def\vs{{\bm{s}}}
\def\vt{{\bm{t}}}
\def\vu{{\bm{u}}}
\def\vv{{\bm{v}}}
\def\vw{{\bm{w}}}
\def\vx{{\bm{x}}}
\def\vy{{\bm{y}}}
\def\vz{{\bm{z}}}

% Elements of vectors
\def\evalpha{{\alpha}}
\def\evbeta{{\beta}}
\def\evepsilon{{\epsilon}}
\def\evlambda{{\lambda}}
\def\evomega{{\omega}}
\def\evmu{{\mu}}
\def\evpsi{{\psi}}
\def\evsigma{{\sigma}}
\def\evtheta{{\theta}}
\def\eva{{a}}
\def\evb{{b}}
\def\evc{{c}}
\def\evd{{d}}
\def\eve{{e}}
\def\evf{{f}}
\def\evg{{g}}
\def\evh{{h}}
\def\evi{{i}}
\def\evj{{j}}
\def\evk{{k}}
\def\evl{{l}}
\def\evm{{m}}
\def\evn{{n}}
\def\evo{{o}}
\def\evp{{p}}
\def\evq{{q}}
\def\evr{{r}}
\def\evs{{s}}
\def\evt{{t}}
\def\evu{{u}}
\def\evv{{v}}
\def\evw{{w}}
\def\evx{{x}}
\def\evy{{y}}
\def\evz{{z}}

% Matrix
\def\mA{{\bm{A}}}
\def\mB{{\bm{B}}}
\def\mC{{\bm{C}}}
\def\mD{{\bm{D}}}
\def\mE{{\bm{E}}}
\def\mF{{\bm{F}}}
\def\mG{{\bm{G}}}
\def\mH{{\bm{H}}}
\def\mI{{\bm{I}}}
\def\mJ{{\bm{J}}}
\def\mK{{\bm{K}}}
\def\mL{{\bm{L}}}
\def\mM{{\bm{M}}}
\def\mN{{\bm{N}}}
\def\mO{{\bm{O}}}
\def\mP{{\bm{P}}}
\def\mQ{{\bm{Q}}}
\def\mR{{\bm{R}}}
\def\mS{{\bm{S}}}
\def\mT{{\bm{T}}}
\def\mU{{\bm{U}}}
\def\mV{{\bm{V}}}
\def\mW{{\bm{W}}}
\def\mX{{\bm{X}}}
\def\mY{{\bm{Y}}}
\def\mZ{{\bm{Z}}}
\def\mBeta{{\bm{\beta}}}
\def\mPhi{{\bm{\Phi}}}
\def\mLambda{{\bm{\Lambda}}}
\def\mSigma{{\bm{\Sigma}}}

% Tensor
\DeclareMathAlphabet{\mathsfit}{\encodingdefault}{\sfdefault}{m}{sl}
\SetMathAlphabet{\mathsfit}{bold}{\encodingdefault}{\sfdefault}{bx}{n}
\newcommand{\tens}[1]{\bm{\mathsfit{#1}}}
\def\tA{{\tens{A}}}
\def\tB{{\tens{B}}}
\def\tC{{\tens{C}}}
\def\tD{{\tens{D}}}
\def\tE{{\tens{E}}}
\def\tF{{\tens{F}}}
\def\tG{{\tens{G}}}
\def\tH{{\tens{H}}}
\def\tI{{\tens{I}}}
\def\tJ{{\tens{J}}}
\def\tK{{\tens{K}}}
\def\tL{{\tens{L}}}
\def\tM{{\tens{M}}}
\def\tN{{\tens{N}}}
\def\tO{{\tens{O}}}
\def\tP{{\tens{P}}}
\def\tQ{{\tens{Q}}}
\def\tR{{\tens{R}}}
\def\tS{{\tens{S}}}
\def\tT{{\tens{T}}}
\def\tU{{\tens{U}}}
\def\tV{{\tens{V}}}
\def\tW{{\tens{W}}}
\def\tX{{\tens{X}}}
\def\tY{{\tens{Y}}}
\def\tZ{{\tens{Z}}}


% Graph
\def\gA{{\mathcal{A}}}
\def\gB{{\mathcal{B}}}
\def\gC{{\mathcal{C}}}
\def\gD{{\mathcal{D}}}
\def\gE{{\mathcal{E}}}
\def\gF{{\mathcal{F}}}
\def\gG{{\mathcal{G}}}
\def\gH{{\mathcal{H}}}
\def\gI{{\mathcal{I}}}
\def\gJ{{\mathcal{J}}}
\def\gK{{\mathcal{K}}}
\def\gL{{\mathcal{L}}}
\def\gM{{\mathcal{M}}}
\def\gN{{\mathcal{N}}}
\def\gO{{\mathcal{O}}}
\def\gP{{\mathcal{P}}}
\def\gQ{{\mathcal{Q}}}
\def\gR{{\mathcal{R}}}
\def\gS{{\mathcal{S}}}
\def\gT{{\mathcal{T}}}
\def\gU{{\mathcal{U}}}
\def\gV{{\mathcal{V}}}
\def\gW{{\mathcal{W}}}
\def\gX{{\mathcal{X}}}
\def\gY{{\mathcal{Y}}}
\def\gZ{{\mathcal{Z}}}

% Sets
\def\sA{{\mathbb{A}}}
\def\sB{{\mathbb{B}}}
\def\sC{{\mathbb{C}}}
\def\sD{{\mathbb{D}}}
% Don't use a set called E, because this would be the same as our symbol
% for expectation.
\def\sF{{\mathbb{F}}}
\def\sG{{\mathbb{G}}}
\def\sH{{\mathbb{H}}}
\def\sI{{\mathbb{I}}}
\def\sJ{{\mathbb{J}}}
\def\sK{{\mathbb{K}}}
\def\sL{{\mathbb{L}}}
\def\sM{{\mathbb{M}}}
\def\sN{{\mathbb{N}}}
\def\sO{{\mathbb{O}}}
\def\sP{{\mathbb{P}}}
\def\sQ{{\mathbb{Q}}}
\def\sR{{\mathbb{R}}}
\def\sS{{\mathbb{S}}}
\def\sT{{\mathbb{T}}}
\def\sU{{\mathbb{U}}}
\def\sV{{\mathbb{V}}}
\def\sW{{\mathbb{W}}}
\def\sX{{\mathbb{X}}}
\def\sY{{\mathbb{Y}}}
\def\sZ{{\mathbb{Z}}}

% Entries of a matrix
\def\emLambda{{\Lambda}}
\def\emA{{A}}
\def\emB{{B}}
\def\emC{{C}}
\def\emD{{D}}
\def\emE{{E}}
\def\emF{{F}}
\def\emG{{G}}
\def\emH{{H}}
\def\emI{{I}}
\def\emJ{{J}}
\def\emK{{K}}
\def\emL{{L}}
\def\emM{{M}}
\def\emN{{N}}
\def\emO{{O}}
\def\emP{{P}}
\def\emQ{{Q}}
\def\emR{{R}}
\def\emS{{S}}
\def\emT{{T}}
\def\emU{{U}}
\def\emV{{V}}
\def\emW{{W}}
\def\emX{{X}}
\def\emY{{Y}}
\def\emZ{{Z}}
\def\emSigma{{\Sigma}}

% entries of a tensor
% Same font as tensor, without \bm wrapper
\newcommand{\etens}[1]{\mathsfit{#1}}
\def\etLambda{{\etens{\Lambda}}}
\def\etA{{\etens{A}}}
\def\etB{{\etens{B}}}
\def\etC{{\etens{C}}}
\def\etD{{\etens{D}}}
\def\etE{{\etens{E}}}
\def\etF{{\etens{F}}}
\def\etG{{\etens{G}}}
\def\etH{{\etens{H}}}
\def\etI{{\etens{I}}}
\def\etJ{{\etens{J}}}
\def\etK{{\etens{K}}}
\def\etL{{\etens{L}}}
\def\etM{{\etens{M}}}
\def\etN{{\etens{N}}}
\def\etO{{\etens{O}}}
\def\etP{{\etens{P}}}
\def\etQ{{\etens{Q}}}
\def\etR{{\etens{R}}}
\def\etS{{\etens{S}}}
\def\etT{{\etens{T}}}
\def\etU{{\etens{U}}}
\def\etV{{\etens{V}}}
\def\etW{{\etens{W}}}
\def\etX{{\etens{X}}}
\def\etY{{\etens{Y}}}
\def\etZ{{\etens{Z}}}

% The true underlying data generating distribution
\newcommand{\pdata}{p_{\rm{data}}}
% The empirical distribution defined by the training set
\newcommand{\ptrain}{\hat{p}_{\rm{data}}}
\newcommand{\Ptrain}{\hat{P}_{\rm{data}}}
% The model distribution
\newcommand{\pmodel}{p_{\rm{model}}}
\newcommand{\Pmodel}{P_{\rm{model}}}
\newcommand{\ptildemodel}{\tilde{p}_{\rm{model}}}
% Stochastic autoencoder distributions
\newcommand{\pencode}{p_{\rm{encoder}}}
\newcommand{\pdecode}{p_{\rm{decoder}}}
\newcommand{\precons}{p_{\rm{reconstruct}}}

\newcommand{\laplace}{\mathrm{Laplace}} % Laplace distribution

\newcommand{\E}{\mathbb{E}}
\newcommand{\Ls}{\mathcal{L}}
\newcommand{\R}{\mathbb{R}}
\newcommand{\emp}{\tilde{p}}
\newcommand{\lr}{\alpha}
\newcommand{\reg}{\lambda}
\newcommand{\rect}{\mathrm{rectifier}}
\newcommand{\softmax}{\mathrm{softmax}}
\newcommand{\sigmoid}{\sigma}
\newcommand{\softplus}{\zeta}
\newcommand{\KL}{D_{\mathrm{KL}}}
\newcommand{\Var}{\mathrm{Var}}
\newcommand{\standarderror}{\mathrm{SE}}
\newcommand{\Cov}{\mathrm{Cov}}
% Wolfram Mathworld says $L^2$ is for function spaces and $\ell^2$ is for vectors
% But then they seem to use $L^2$ for vectors throughout the site, and so does
% wikipedia.
\newcommand{\normlzero}{L^0}
\newcommand{\normlone}{L^1}
\newcommand{\normltwo}{L^2}
\newcommand{\normlp}{L^p}
\newcommand{\normmax}{L^\infty}

\newcommand{\parents}{Pa} % See usage in notation.tex. Chosen to match Daphne's book.

\DeclareMathOperator*{\argmax}{arg\,max}
\DeclareMathOperator*{\argmin}{arg\,min}

\DeclareMathOperator{\sign}{sign}
\DeclareMathOperator{\Tr}{Tr}
\let\ab\allowbreak


\usepackage{hyperref}
\usepackage{url}


\title{Which spherical CNN should you use?\\ DeepSphere V2}

% Authors must not appear in the submitted version. They should be hidden
% as long as the \iclrfinalcopy macro remains commented out below.
% Non-anonymous submissions will be rejected without review. 
\author{Michaël Defferrard, Martino Milani, Frédérick Gusset,  Nathanaël Perraudin \thanks{ Use footnote for providing further information about author (webpage, alternative address)---\emph{not} for acknowledging
funding agencies.  Funding acknowledgements go at the end of the paper.} \\
Department of Computer Science\\
Cranberry-Lemon University\\
Pittsburgh, PA 15213, USA \\
\texttt{\{hippo,brain,jen\}@cs.cranberry-lemon.edu} \\
\And
Ji Q. Ren \& Yevgeny LeNet \\
Department of Computational Neuroscience \\
University of the Witwatersrand \\
Joburg, South Africa \\
\texttt{\{robot,net\}@wits.ac.za} \\
\AND
Coauthor \\
Affiliation \\
Address \\
\texttt{email}
}

% The \author macro works with any number of authors. There are two commands
% used to separate the names and addresses of multiple authors: \And and \AND.
%
% Using \And between authors leaves it to \LaTeX{} to determine where to break
% the lines. Using \AND forces a linebreak at that point. So, if \LaTeX{}
% puts 3 of 4 authors names on the first line, and the last on the second
% line, try using \AND instead of \And before the third author name.

\newcommand{\fix}{\marginpar{FIX}}
\newcommand{\new}{\marginpar{NEW}}

%\iclrfinalcopy % Uncomment for camera-ready version, but NOT for submission.
\begin{document}

\maketitle

\begin{abstract}

story: there are tradeoffs when designing a spherical CNN

* method development: tool that solve a need (balance btw the desiderata: equiv vs cost)\\
* research question: anisotropy (highlight from the start or leave it as a dangling question in the end)\\

contributions:\\
* theory => convergence (better graph)\\
* experiments on relevant problems (not spherical MNIST) => check the desiderata\\
  * surprising: anisotropy doesn't seem useful\\

\end{abstract}

\section{Introduction [2 pages]}
Get inspiration from DeepSphere V1\\

Why are spherical CNNs important? What are the application?\\
See DeepSphere workshop paper\\

Desiderata:\\
* respect the domain\\
  * rotational equivariance\\
  * deformation (e.g., on the icosahedron for gauge)\\
* powerful / generality => isotropic vs anisotropic\\
* scale: computational cost \& memory usage (feature maps)\\
* flexible (samplings, irregular), simplicity (implementation)\\

\begin{table}[h!]
    \centering
    \begin{tabular}{l|c|c|c|c}
         & Scale & Generality& equivariance & Flexibility \\
         \hline
        Cohen & & & &  \\
         \hline
        Gauge & & & &  \\
         \hline
        Esteves & & & &  \\
         \hline
        Jiang & & & &  \\
         \hline
        2D CNN (cubed sphere) & & & &  \\
         \hline
        Ours & & & &  \\
    \end{tabular}
    \caption{Caption}
    \label{tab:my_label}
\end{table}

On the one side, we have Cohen which is computationally very expensive but perfectly equivariant, on the other we have cube sphere which computationl very good but not equivariant at all. Other methods are tradedoff between these two. What is a good tradeoff? We think that DeepSphere is

Present the different spherical CNNs with pro and cons => mostly scale vs generality\\
* 2D CNN => doesn't respect geometry and equivariance\\
* Cohen spherical CNN => most general, but doesn't scale\\
* Cohen gauge => fixes scale, at the price of deformation (sphere => icosahedron)\\
* Esteves => ? \\
* Jiang => need a global coordinate system (ok for planets, not projections like cosmo)\\

\section{Method [1 pages]}

* HEALPix: improvement from DeepSphere v1 \cite{perraudin2019deepsphere}
* equiangular: Renata \& Pascal

\section{Rotation equivariance [2.5 pages]}

\subsection{Link with the Laplace-Beltrami Operator}

\subsection{HEALPix: convergence theorem}

\subsection{Equiangular}

explain rapidly Renata \& Pascal

\subsection{Equivariance error}

* define a measure of equiv error\\
* show convergence empirically (for Nside up to 1024) \\
* improved upon V1\\
* no difference in practice (see experiment xx) => NNs are resilient to equiv error\\

\section{Experiments [2 pages]}

Show: \\
* meet the desiderata \\
* DeepSphere V1 and V2 are equivalent in practice \\
* anisotropy doesn't help \\

\subsection{3D object recognition}

* same perf as other spherical CNNs, but 40 times faster \\
* compare the two samplings => not much difference \\

\begin{table}[ht]
    \centering
    \begin{tabular}{l|c c r r}
        \multicolumn{1}{l}{} & \multicolumn{2}{c}{performance} & \multicolumn{2}{c}{speed}\\
        \cmidrule(lr){2-3} \cmidrule(lr){5-6}
        \multicolumn{1}{l}{Method} & Accuracy & F1-score & inference & training \\ \hline
        Cohen \emph{s2cnn} & - & - & 38ms & 50h\\
        % Cohen \emph{s2cnn\_simple} & 78.59 & 78.85 & 400k & 12ms & 32h\\
        Esteves \emph{sphericalcnn} & 79.18 & 79.36 & 9.8ms & 2h52\\ \hline
        Deepsphere \emph{Equiangular} & 79.25 & 79.36 & 0.98ms & 2h33m \\
        Deepsphere \emph{HEALPix} & 80.42 & 80.65 & 1.0ms & 48m
    \end{tabular}
    \caption{Performance of different models}
    \label{tab:SHREC17_class}
\end{table}

\begin{table}[ht]
    \centering
    \begin{tabular}{l|c c c c|c c c c}
     & \multicolumn{4}{c|}{micro (label average)} & \multicolumn{4}{c}{macro (instance average)} \\
    Method & P@N & R@N & F1@N & mAP & P@N & R@N & F1@N & mAP \\ \hline
    Cohen \emph{s2cnn} & 0.701 & 0.711 & 0.699 & 0.676 & - & - & - & - \\
    % Cohen \emph{s2cnn\_simple} & 0.704 & 0.701 & 0.696 & 0.665 & 0.430 & 0.480 & 0.429 & 0.385\\
    Esteves \emph{sphericalcnn} & 0.717 & 0.737 & - & 0.685 & 0.450 & 0.550 & - & 0.444\\ \hline
    Deepsphere \emph{Equiangular} & 0.709 & 0.700 & 0.698 & 0.665 & 0.439 & 0.489 & 0.439 & 0.403 \\
    Deepsphere \emph{HEALPix} & 0.725 & 0.717 & 0.715 & 0.686 & 0.475 & 0.508 & 0.468 & 0.428
    \end{tabular}
    \caption{Performance of different models over SHREC17 perturbed dataset as a retrieval task, using the official script of the competition.}
    \label{tab:SHREC17_retriev}
\end{table}
* add MN40 too?
* add improved graph results too?

\subsection{cosmo}

* other sphercial CNNs cannot scale to 10M pixels (tested on 10k at most) \\
* redo the experiment with the optimal graph (cannot just copy the old results) and compare \\

\subsection{Climate event detection}

* small: better perf than gauge and Jiang. due to icosahedron distortion? to be confirmed\\
* cannot compare with Mudigonda (16 vs 1 input channel)\\
* full: we scale (50B pixels, 20TB) => lead to better perf?\\

\begin{table}[!ht]
\begin{tabular}{l|c c c c c c c c c}
        \multicolumn{1}{l}{} & \multicolumn{4}{c}{Accuracy} & \multicolumn{3}{c}{Average Precision} & \multicolumn{2}{c}{Speed}\\
        \cmidrule(lr){2-5}\cmidrule(lr){6-8}\cmidrule(lr){9-10}
        \multicolumn{1}{l}{Method} & BG & TC & AR & mean & TC & AR & mean & inference & training \\ \hline
        %Mudigonda et al. & 97 & 74 & 65 & 78.67 & - & - & - \\
        Jiang et al. & 97 & 94 & 93 & 94.67 & - & - & - & - & -\\
        Jiang (bsize 64) & 95.2 & 93.9 & 95.7 & 94.95 & 11.08 & 65.21 & 38.41 & 10.2 ms & 10h14m\\
        % 328'339
        Cohen et al. (S2R) & 97.3 & 97.8 & 97.3 & 97.5 & - & -& 68.6\\
        Cohen et al. (R2R) & 97.4 & 97.9 & 97.8 & 97.7 & - & -& 75.9\\ \hline
        DS (ico) & $98.2\pm 0.5$ & $97.4\pm 1.1$ & $97.7\pm 0.7$ & $97.8\pm 0.3$ & $58.88\pm 3.17$ & $95.41\pm 1.51$ & $77.15\pm 1.94$ & 25 ms & 25h \\
        %DS (non-weighted) & 99.8 & 73.3 & 90.2 & 87.81 & 79.29 & 97.93 & 88.61 & 30ms & 3h50m\\
        DS (non-weighted) & 99.6 & 66.6 & 96.1 & 87.41 & 83.66 & 97.57 & 90.62 & 30ms & 12h20m\\ \hline
        % 12'926'432
        DeepSphere (equi non-weighted) & 99.9 & 31.3 & 75.2 & 68.80 & 55.53 & 94.85 & 75.19 & 567 ms & 48h
    \end{tabular}
    \caption{Climate pattern segmentation accuracy (\%) for BG, TC
and AR classes plus mean accuracy, average precision for positive classes and mean as well.}
\end{table}

trade off between AP and accuracy in the case of TC class, it seems.

\subsection{GHCN}

* task is artificial, but it shows DeepSphere's flexibility\\
* structure (looking around) help for prediction\\

dense regression
\begin{table}[!ht]
    \centering
    \begin{tabular}{c|ccc}
        order & MSE & MAE & R2 \\ \hline 
        0 & 10.88 & 2.42 & 0.896\\
        1 & 8.91 & 2.20 & 0.906\\
        4 & 8.20 & 2.11 & 0.919\\
        9 & 8.38 & 2.12 & 0.915\\
    \end{tabular}
    \caption{The task is to find the temperature for the day T, knowing the temperature of the 5 previous days. Evolution of performance in respect of the order of the filter}
    \label{tab:future_results}
    ~\\[1cm]
    \begin{tabular}{c|ccc}
        order & MSE & MAE & R2 \\ \hline 
        1 & 134.01 & 10.51 & -0.319\\
        4 & 129.48 & 10.31 & -0.274\\
        9 & 141.68 & 10.82 & -0.389\\
    \end{tabular}
    \caption{Predicting the present day (T-1) instead of the future day T}
    \label{tab:present_results}
\end{table}

global regression
\begin{table}[!ht]
    \centering
    \begin{tabular}{c|ccc}
        order & MSE & MAE & R2 \\ \hline 
        0 & 0.58 & 0.42 & -0.92\\
        4 & 0.50 & 0.18 & 0.597\\
    \end{tabular}
    \caption{Find day in year using precipitation}
    \label{tab:future_results}
    ~\\[1cm]
    \begin{tabular}{c|ccc}
        order & MSE & MAE & R2 \\ \hline 
        0 & 0.10 & 0.10 & 0.881\\
        4 & - & - & -\\
    \end{tabular}
    \caption{Using temperature feature that are already periodical}
    \label{tab:present_results}
\end{table}

\section{Conclusion [0.5 pages]}

* DeepSphere seems to be in the sweet spot of tradeoffs. Recall the desiderata.\\
  * respect geometry: No need for a very precise equivariance, i.e. a very good graph...\\
  * Not the most general (restricted to isotropic filters), but didn't hurt perf in any experiment\\
  * scales linearly => cannot do better ($n^1.25$ for best HEALPix, linear for not optimal)\\
  * flexible => independent of the sampling\\

future work:\\
* when anisotropy useful? (research question)\\
* irregular sampling while respecting the geometry\\
* beyond the sphere, any manifold\\

\newpage
\subsubsection*{Author Contributions}
Left blank for anonymity reason
% If you'd like to, you may include  a section for author contributions as is done
% in many journals. This is optional and at the discretion of the authors.

\subsubsection*{Acknowledgments}
Left blank for anonymity reason
% Use unnumbered third level headings for the acknowledgments. All
% acknowledgments, including those to funding agencies, go at the end of the paper.

\bibliography{references}
\bibliographystyle{iclr2020_conference}

\newpage

{\LARGE \sc {Title of the paper: Supplementary Material}}
\appendix

\section{Proof of theorem}


\section{Experiments details}
* Network architecture and details
\subsection{3D object Recognition}
HEALPix sampling, Nside = 32

\subsection{cosmo}
HEALPix sampling, Nside = 1024, with improved graph

\subsection{climate event detection}
icosahedron sampling, level-5 resolution

\subsection{GHCN}
non-regular sampling

\end{document}